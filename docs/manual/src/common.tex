\begin{titlepage}
  \begin{center}

  {\Huge FIFO\_STIMULATOR}

  \vspace{25mm}

  \includegraphics[width=0.90\textwidth,height=\textheight,keepaspectratio]{img/AFRL.png}

  \vspace{25mm}

  \today

  \vspace{15mm}

  {\Large Jay Convertino}

  \end{center}
\end{titlepage}

\tableofcontents

\newpage

\section{Usage}

\subsection{Introduction}

\par
This core contains two modules. A writer, and reader that should be placed on the
output, and input of the device under test. This will stream data through till
is has read all data.

\subsection{Dependencies}

\par
The following are the dependencies of the cores.

\begin{itemize}
  \item fusesoc 2.X
  \item iverilog (simulation)
  \item cocotb (simulation)
\end{itemize}

\input{src/fusesoc/depend_fusesoc_info.tex}

\section{Architecture}
\par
The project contains two modules write\_fifo\_stimulus and read\_fifo\_stimulus. The read\_fifo\_stimulus is used to take
input data from the wr\_fifo interface (input) and write it to a file. Essentially it goes DUT\_RD\_FIFO to read\_fifo\_stimulus(WR\_FIFO).
write\_fifo\_stimulus is used to read a file and push that data to the rd\_fifo interface (output). Essentially it goes
write\_fifo\_stimulus(WR\_FIFO) to DUT\_RD\_FIFO.
\par
This core uses a custom library for reading and writing files called vpi\_binary\_file\_io. This library provides multithreaded file reads
using a ring buffer between processes. The core will also puncture data according to its bit type. X/Z values are tossed if they are contained
in a byte.

\begin{itemize}
  \item \textbf{tm\_stim\_fifo} Contains two modules write\_fifo\_stimulus and read\_fifo\_stimulus.
\end{itemize}

Please see \ref{Module Documentation} for more information per target.

\section{Building}

\par
The all FIFO stimulator modules are written in Verilog 2001. They should synthesize in any modern FPGA software. The core comes as a fusesoc packaged core and can be
included in any other core. Be sure to make sure you have meet the dependencies listed in the previous section.

\subsection{fusesoc}
\par
Fusesoc is a system for building FPGA software without relying on the internal project management of the tool. Avoiding vendor lock in to Vivado or Quartus.
These cores, when included in a project, can be easily integrated and targets created based upon the end developer needs. The core by itself is not a part of
a system and should be integrated into a fusesoc based system. Simulations are setup to use fusesoc and are a part of its targets.

\subsection{Source Files}

\subsubsection{fusesoc\_info File List}
\begin{itemize}
\item src
	\begin{itemize}
	\item {'src/tm\_stim\_fifo.v': {'file\_type': 'verilogSource'}}
	\end{itemize}
\item tb
	\begin{itemize}
	\item {'tb/tb\_fifo.v': {'file\_type': 'verilogSource'}}
	\end{itemize}
\end{itemize}


\subsection{Targets} \label{targets}

\subsubsection{fusesoc\_info Targets}
\begin{itemize}
\item default
	\begin{itemize}
	\item[$\space$] Info: Default for simulation fileset.
	\end{itemize}
\item sim
	\begin{itemize}
	\item[$\space$] Info: Default for icarus simulation.
	\end{itemize}
\item sim\_rand\_data
	\begin{itemize}
	\item[$\space$] Info: Random data input.
	\end{itemize}
\item sim\_rand\_full\_rand\_data
	\begin{itemize}
	\item[$\space$] Info: Random data, random ready input.
	\end{itemize}
\item sim\_8bit\_count\_data
	\begin{itemize}
	\item[$\space$] Info: Counter data input.
	\end{itemize}
\item sim\_rand\_full\_8bit\_count\_data
	\begin{itemize}
	\item[$\space$] Info: Counter data, random ready input.
	\end{itemize}
\end{itemize}


\subsection{Directory Guide}

\par
Below highlights important folders from the root of the directory.

\begin{enumerate}
  \item \textbf{docs} Contains all documentation related to this project.
    \begin{itemize}
      \item \textbf{manual} Contains user manual and github page that are generated from the latex sources.
    \end{itemize}
  \item \textbf{src} Contains source files for fifo\_stimulator.
  \item \textbf{tb} Contains test bench files.
\end{enumerate}

\newpage

\section{Simulation}
\par
There is no simulation at the moment. Maybe a future addition?

\newpage

\section{Module Documentation} \label{Module Documentation}

\par
There project has multiple modules. The targets are the top system wrappers.

\begin{itemize}
\item \textbf{tm\_stim\_fifo}
\item \textbf{tb\_fifo}
\end{itemize}
The next sections document the module in great detail.

